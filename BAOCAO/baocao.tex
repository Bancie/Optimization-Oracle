\documentclass{beamer}
\usepackage[utf8]{vietnam}
\usepackage{graphicx}
\usepackage{booktabs}
\usetheme{Warsaw}
\newtheorem{dn}{Định nghĩa}[section]
\newtheorem{dl}{Định lý}[section]
\newtheorem{tc}{Tính chất}[section]
\newtheorem{hq}{Hệ quả}[section]
\newtheorem{bd}{Bổ đề}[section]
\newtheorem{md}{Mệnh đề}[section]
\newtheorem{vd}{Ví dụ}[section]
\newtheorem{nx}{Nhận xét}[section]
\newcommand{\dom}{\text{{\rm dom}}}
\newcommand{\epi}{\text{{\rm epi}}}
\newcommand{\Min}{\text{{\rm Min}}}
\setbeamertemplate{theorems}[numbered]
\setbeamertemplate{definitions}[numbered]
\setbeamertemplate{footline}[frame number]
\usepackage{algorithm}
\usepackage{color}
\usepackage{algorithmic}
\usepackage{footmisc}
%\usepackage{enumitem}
\usepackage{indentfirst} 
\usepackage{comment}
\renewcommand{\thefootnote}{\arabic{footnote}}
\usefonttheme{professionalfonts}
\setbeamercolor{normal text}{bg=white,fg=black}
\renewcommand{\thefootnote}{\arabic{footnote}}
%kt
\mode<presentation>
{
 \usetheme{Darmstadt}
%\usetheme{Rochester}
}
\beamertemplatetransparentcoveredhigh

\begin{document}
\title[]{\fontsize{13pt}{10pt}\selectfont {\bf \LARGE   Phương pháp giải bài toán \\Tối ưu tuyến tính nguyên}\\
------------------------------------------

{\small Hướng dẫn: PGS.TS. Tạ Quang Sơn}} 
\author[]{\bf Thực hiện : ĐỖ NGỌC MINH THƯ \& NGUYỄN CHÍ BẰNG \\
Sinh viên lớp: DTU1221, Khóa: 22}
\institute[Báo cáo luận văn thạc sĩ]{\fontsize{2pt}{2pt}}%
\small{\date{\today}}
\begin{frame}
\begin{center}
{\fontsize{8pt}{8pt}\selectfont \bf{ỦY BAN NHÂN DÂN THÀNH PHỐ HỒ CHÍ MINH\\
TRƯỜNG ĐẠI HỌC SÀI GÒN}}
\end{center}
\begin{center}
\end{center}

\begin{center}
{\fontsize{10pt}{6pt}\selectfont \bf{BÁO CÁO ĐỀ CƯƠNG NGHIÊN CỨU KHOA HỌC\\
NGÀNH: TOÁN ỨNG DỤNG}}
\end{center}
\titlepage
\end{frame}
\begin{frame}
    \frametitle{NỘI DUNG BÁO CÁO}
    \tableofcontents
\end{frame}

\section{Mục đích nghiên cứu}

\begin{frame}{Mục đích nghiên cứu}
    Tối ưu tuyến tính là một nội dung quan trọng trong chương trình đào tạo Cử nhân Toán ứng dụng. Lý thuyết về việc giải bài toán tối ưu tuyến tính đã được cung cấp cho sinh viên. Tuy vậy, có nhiều bài toán tối ưu cần được giải với nghiệm nguyên. Chẳng hạn như:
    \begin{itemize}
    \item Bài toán tối ưu nhân lực vận chuyển hàng hóa.
    \item Bài toán tối ưu áp dụng trong tin học.
    \end{itemize}
\end{frame}

\begin{frame}{Bài toán dẫn nhập}
    Một công ty cần sản xuất 2 dòng xe máy để đưa ra thị trường. Biết rằng với dòng xe thứ nhất cần 8 \$ cho phụ tùng và 5 \$ cho chi phí thuê nhân công. Đối với dòng xe thứ hai chi phí lần lượt chia cho phụ tùng và thuê nhân công là 6 \$ và 7 \$ cho mỗi sản phẩm. Được biết tổng vốn đầu tư của công ty sẽ bỏ ra theo kế hoạch là 1200 \$ cho phụ tùng và 2000 \$ cho nhân công. Hãy tìm ra phương án đầu tư để đạt được lợi nhuận cao nhất với lợi nhận của dòng xe thứ nhất là 10 \$ 1 sản phẩm và dòng thứ hai là 7 \$.
\end{frame}
\section{Nội dung nghiên cứu}
\begin{frame}{Nội dung nghiên cứu}
    \begin{itemize}
    \item Hệ thống lại cơ sở lý thuyết và phương pháp giải các bài toán Quy hoạch tuyến tính.
    \item Từ đó mở rộng để tìm nghiệm nguyên cho bài toán, thông qua 2 phương pháp:
    \begin{itemize}
    \item Phương pháp Gomory.
    \item Phương pháp Land-Doig.
    \end{itemize}
    \end{itemize}
\end{frame}
\section{Dự kiến luận văn}
\begin{frame}{Dự kiến luận văn}
    \begin{itemize}
    \item Chương 1:  Bao gồm các kiến thức chuẩn bị, nội dung có liên quan đến
    một số kiến thức cơ bản của quy hoạch tuyến tính để dùng
    làm cơ sở nghiên cứu về các phương pháp giải của bài toán quy hoạch nguyên.
    \item Chương 2: Tìm hiểu về các phương pháp và thuật giải giúp giải quyết bài toán quy hoạch nguyên bằng phương pháp Gomory và phương pháp Land-Doig.
    \item Chương 3: Nghiên cứu xoay quanh những vấn đề và ứng dụng có liên quan đến bài toán Quy hoạch nguyên. Trong đó có 3 bài toán quan trọng cần quan tâm:
    \begin{itemize}
        \item Bài toán cái túi.
        \item Bài toán người du lịch.
        \item Bài toán gia công chi tiết máy. 
    \end{itemize}
    \end{itemize}   
\end{frame}
\section{Tổ chức và phân công}
\begin{frame}{Tổ chức và phân công}
    \begin{table}
        \begin{tabular}{|c|c|}
            \hline
            Họ và tên & Nội dung \\
            \hline \hline
            Đỗ Ngọc Minh Thư & Nghiên cứu về tối ưu tuyến tính nguyên. \\
            Nguyễn Chí Bằng & Nghiên cứu về tối ưu tuyến tính nguyên. \\
            \hline
        \end{tabular}
    \end{table}
\end{frame}
\section{Tiến độ thực hiện}
\begin{frame}[shrink=25]
    \frametitle{Tiến độ thực hiện}
    \vspace{2cm}
    \begin{table}
        \begin{tabular}{|c|c|c|}
            \hline
            Công việc thực hiện & Thời gian & Người thực hiện \\
            \hline \hline
            Nghiên cứu tối ưu tuyến tính nguyên. & 1/8/2023 - 1/4/2024 & Đỗ Ngọc Minh Thư \\
            Nghiên cứu tối ưu tuyến tính nguyên. & 1/8/2023 - 1/4/2024 & Nguyễn Chí Bằng \\
            \hline
        \end{tabular}
    \end{table}
\end{frame}
\section{Tài liệu dùng cho nghiên cứu}
\begin{frame}[allowframebreaks]
    \frametitle{Tài liệu dùng cho nghiên cứu}
    \bibliographystyle{amsalpha}
    \bibliography{tailieunc}
\end{frame}
\end{document}

