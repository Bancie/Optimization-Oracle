\documentclass{beamer}
\usepackage[utf8]{vietnam}
\usetheme{Warsaw}
\title{TỐI ƯU TUYẾN TÍNH NGUYÊN}
\subtitle{PGS. TS Tạ Quang Sơn}
\author{Đỗ Ngọc Minh Thư \\ Nguyễn Chí Bằng}
\institute{Khoa Toán Ứng dụng \\ Trường Đại học Sài Gòn}
\date{\today}

\begin{document}

\begin{frame}
\titlepage
\end{frame}

\begin{frame}
\begin{center}
{\huge BUỔI XÉT DUYỆT ĐỀ TÀI}
\\[2\baselineskip]
\textbf{THÀNH VIÊN HỘI ĐỒNG}

TS. Nguyễn Văn Huấn

PGS. TS. Võ Hoàng Hưng

TS. Lương Duy Bình
\end{center}
\end{frame}


\begin{frame}
    \frametitle{Outline}
    \tableofcontents
\end{frame}

\section{Section 1}
\subsection{sub a}

\begin{frame}{BÀN CÓ 5 CHỖ NGỒI}
    Huy, Hiền, Quang, Đại, Bảy – họ là năm người bạn với năm cá tính và hoàn cảnh khác nhau cùng chung trong một lớp học. Những trò nghịch ngợm trẻ con đôi khi gây ra mâu thuẫn, nhưng trên tất cả đó là những đứa trẻ ham học, giàu lòng nhân ái và biết quan tâm đến bạn bè. Cảm thông với hoàn cảnh của nhau, từng người nghĩ ra cách giúp đỡ bạn theo khả năng của mình để tình bạn ấy lớn dần theo năm tháng.
\end{frame}

\begin{frame}{BỒ CÂU KHÔNG ĐƯA THƯ}
    Câu chuyện bắt đầu từ lá thư làm quen để trong hộc bàn của Thục – một cô nữ sinh trong bộ ba Xuyến, Thục, Cúc Hương. Lá thư chân tình đã thu hút sự tò mò của cả ba, và họ bị cuốn hút vào trò chơi với người giấu mặt, dần hồi kéo theo Phán củi – anh chàng xấu xí, vụng về của lớp – làm quân sư và giúp xướng hoạ thơ. Cuộc truy tìm dẫn mọi người đến nhiều hiểu lầm tai hại và cả những bất ngờ thú vị.
\end{frame}

\subsection{sub b}

\begin{frame}{ BONG BÓNG LÊN TRỜI}
    Vì hoàn cảnh, Thường phải giúp mẹ bằng nghề bán kẹo kéo ngoài giờ học và làm quen với cuộc sống trên đường phố. Ở đó, cậu đánh bạn với những người nghèo và hiểu thêm nhiều điều không có trong sách vở, nhà trường. Cô bé bán bong bóng Tài Khôn hồn nhiên và nhiều ước mơ cũng quan tâm giúp đỡ Thường thoát khỏi mặc cảm nhà nghèo và sống tự tin.
\end{frame}
\section{Section 2}

\begin{frame}{ BUỔI CHIỀU WINDOWS}
    Truyện lấy khung cảnh là phòng thu máy vi tính của một công ty trách nhiệm hữu hạn – nơi bộ ba nữ sinh nghịch ngợm Xuyến, Thục, Cúc Hương bạo gan xin đến làm nhân viên vi tính (tạm thời trong thời gian hè) khi chưa biết tí gì về tin học. Truyện còn mô tả những mộng mơ đầu đời của tuổi mới lớn với những tình tiết vui nhộn.
\end{frame}
\end{document}

