\documentclass{beamer}
\mode<presentation>
\setbeamertemplate{bibliography item}{}
\usepackage[utf8]{vietnam}
\usepackage{beamerthemesplit}
\usepackage{graphicx}
\usepackage{booktabs}
\usepackage{amsmath}
\usepackage{pgfplotstable}
\usepackage{textpos}
\usepackage{pgfplots}
\usepackage{tikz}
\usepackage{hyperref}
\usepackage{caption}
\usetikzlibrary {datavisualization} 
\pgfplotsset{compat=1.18, width = 7cm}
\usetikzlibrary{patterns}
\setbeamertemplate{bibliography item}[text]
\usetheme{CambridgeUS} % AnnArbor, Ilmenau, Darmstadt, Dresden, CambridgeUS, Frankfurt, Singapore
\newtheorem{dn}{Định nghĩa}[section]
\newtheorem{dl}{Định lý}[section]
\newtheorem{tc}{Tính chất}[section]
\newtheorem{hq}{Hệ quả}[section]
\newtheorem{bd}{Bổ đề}[section]
\newtheorem{md}{Mệnh đề}[section]
\newtheorem{vd}{Ví dụ}[section]
\newtheorem{nx}{Nhận xét}[section]
\newcommand{\dom}{\text{{\rm dom}}}
\newcommand{\epi}{\text{{\rm epi}}}
\newcommand{\Min}{\text{{\rm Min}}}
\setbeamertemplate{theorems}[numbered]
\setbeamertemplate{definitions}[numbered]
\setbeamertemplate{footline}[frame number]
\usepackage{algorithm}
\usepackage{color}
\usepackage{algorithmic}
\usepackage{footmisc}
\usepackage{indentfirst} 
\usepackage{comment}
\renewcommand{\thefootnote}{\arabic{footnote}}
\usefonttheme{professionalfonts}
\setbeamercolor{normal text}{bg=white,fg=black}
\renewcommand{\thefootnote}{\arabic{footnote}}
\beamertemplatetransparentcoveredhigh
\title[]{\fontsize{13pt}{10pt}\selectfont {\bf \Large ỨNG DỤNG VÀ CÁCH TẠO \\ THANG ĐO LIKERT}\\}
\author[]{ Nguyễn Chí Bằng \\ 3122480004 \\ DTU1221}
\small{\date{\today}}

\begin{document}

\begin{frame}
\titlepage
\end{frame}

\begin{frame}{TÓM TẮT}
    Trong phần này chúng ta sẽ tìm hiểu về:
    \begin{itemize}
    \item Ứng dụng của thang đo Likert
    \item Hướng dẫn chi tiết cách tạo thang đo Likert.
    \end{itemize}
    \end{frame}    

\begin{frame}
    \frametitle{NỘI DUNG}
    \tableofcontents
\end{frame}    

\section{Ứng dụng và ví dụ của thang đo Likert}
\begin{frame}{Ứng dụng và ví dụ của thang đo Likert}
\begin{itemize}
    \item Chúng ta sẽ khám phá ứng dụng đa dạng của thang đo Likert từ nghiên cứu khoa học đến quản lý tổ chức và đo độ hài lòng của khách hàng. 
    \item Thang đo này không chỉ là công cụ đo lường mà còn là cầu nối quan trọng giữa nhà nghiên cứu và cộng đồng.
    \item Thang đo Likert là một phương tiện đa dạng và hiệu quả trong việc thu thập ý kiến và đánh giá. Ví dụ trong lĩnh vực:
    \begin{itemize}
    \item Khoa học thần kinh.
    \item Dịch vụ.
\end{itemize}
\end{itemize}
\end{frame}

\begin{frame}{Thang đo Likert trong khoa học thần kinh}
    \begin{itemize}
    \item Kết nối qua điện thoại thông minh và mạng xã hội đã trở thành không thể thiếu trong đời sống hàng ngày. Mặc dù nhiều người có trải nghiệm tích cực, sử dụng quá mức có thể ảnh hưởng tiêu cực đến sức khỏe tâm thần, đặc biệt là đối với giới trẻ. Nghiên cứu về lĩnh vực này đang ở giai đoạn đầu, nhưng thang đo Likert giúp đánh giá tác động của mạng xã hội.
    \end{itemize}
\end{frame}

\begin{frame}
    \pgfplotstableread{
Place 1 2 3 4 5 6
\textbf{Facebook} 2 8 8 6 8 67
\textbf{TikTok} 16 32 9 5 4 33
\textbf{YouTube} 19 41 17 12 6 5
\textbf{Instagram} 10 27 12 7 5 38
\textbf{Snapchat} 15 29 7 3 5 41
}\testdata
\begin{figure}
\centering
\begin{tikzpicture}
    \begin{axis}[
                legend cell align=left,
                legend columns=3Z,
                legend style = {at={(0.5, -0.23)}, anchor=north, inner sep=3pt, style={column sep=0.15cm}},
                xbar stacked,
                xmin=0,
                xmax=100,
                xticklabel=\pgfmathprintnumber{\tick}\,$\%$,
                ytick=data,
                yticklabels from table={\testdata}{Place}
                ]
        %add descrip
        \addlegendimage{empty legend}
        \addlegendimage{empty legend}
        \addlegendimage{empty legend}
        \addlegendentry{}
        \addlegendentry{\textbf{Mức độ sử dụng}}
        \addlegendentry{}
        %add plot
        \addplot [fill=blue] table [x=1, meta=Place ,y expr=\coordindex] {\testdata};
        \addlegendentry{Liên tục}

        \addplot [fill=blue!70] table [x=2, meta=Place ,y expr=\coordindex] {\testdata};
        \addlegendentry{Thỉnh thoảng}

        \addplot [fill=blue!60] table [x=3, meta=Place ,y expr=\coordindex] {\testdata};
        \addlegendentry{1 lần/ngày}

        \addplot [fill=blue!50] table [x=4, meta=Place ,y expr=\coordindex] {\testdata};
        \addlegendentry{vài lần/tuần}

        \addplot [fill=blue!40] table [x=5, meta=Place ,y expr=\coordindex] {\testdata};
        \addlegendentry{Ít sử dụng}

        \addplot [fill=blue!30] table [x=6, meta=Place ,y expr=\coordindex] {\testdata};
        \addlegendentry{Không sử dụng}
    \end{axis}
\end{tikzpicture}
\caption{\centering $\%$ thanh thiếu niên Hoa Kỳ nói rằng họ truy cập hoặc sử dụng từng trang web hoặc ứng dụng trên.}
\label{mxh}
\end{figure}
\end{frame}
\end{document}